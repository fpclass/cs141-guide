\section{Exam (60\%)}

The exam is worth 60\% of the module and takes place in Term 3. You will have to answer any four out of six questions. Each question is worth 25 marks. Past exam papers as well as a sample exam paper are available on the module website. Note that since the exam in 2020/21 will be online, the format of some questions may change accordingly and an updated sample paper will be made available reflecting this. In addition to the past papers on the module website, you may also be able to find CS256 exam papers from before 2017/18, but note that their format and content are significantly different and I would not recommend those for revision.

In general, I prefer to set exam questions which require you to use your understanding of functional programming and Haskell to solve problems of varying difficulties. There are unlikely to be any bookwork questions and there will be no lengthy essay questions for you to answer. %To lessen the need for memorisation, you are permitted to take this guide into the exam with you. 
A reference of the Haskell standard library which includes the types and simplified definitions of many useful functions may be found in \Cref{ch:prelude}. % You are permitted to make reasonable annotations: \emph{e.g.} short clarifying comments in the margins, highlights, or bookmarks. However, there should not be any lengthy or substantial amounts of \emph{e.g.} text or code added. The invigilators may inspect your guide and confiscate it if your annotations exceed what we would consider reasonable. Note also that the guide is not required to complete the exam, so although we would try and provide you with a replacement if you do not have your original copy for whatever reason, we may not have any spares and you would not be entitled to a replacement.

There will be at least three revision lectures in Term 3, the dates of which are to be confirmed. I typically discuss some tips and tricks that allow you to answer questions more easily, so you are strongly encouraged to attend. You are also welcome to suggest topics if you wish.