\section{Type classes}
\topics{Type classes and type class instances.}

These exercises are about type classes in Haskell. You can obtain the skeleton code by cloning the corresponding repository from GitHub:
\begin{minted}{bash}
$ git clone https://github.com/fpclass/lab-type-classes
\end{minted}

Type classes in Haskell are used to constrain parametric polymorphism. This is useful because ordinarily, when we have a type variable, we know nothing about the type that may be instantiated for it. For example, in the definition of the identity function all we can do with \texttt{\small x} is return it:
\begin{minted}{haskell}
id :: a -> a
id x = x
\end{minted} 
We cannot do anything else with \texttt{\small x} since we know nothing about it. For example, if we wanted to define a function which appends some text to the result of \texttt{\small show}ing a value, we could not write:
\begin{minted}{haskell}
showAndAppend :: a -> String 
showAndAppend x = "Value of x is:" ++ show x
\end{minted}
This is a type error because we do not know that \texttt{\small show} is defined for whatever type \texttt{\small a} gets instantiated with when we use this \texttt{\small showAndAppend} function. Instead, we need to constrain \texttt{\small a} to only those types which implement the \haskellIn{Show} type class (whose interface contains the \texttt{\small show} function):
\begin{minted}{haskell}
showAndAppend :: Show a => a -> String 
showAndAppend x = "Value of x is:" ++ show x
\end{minted}
This now compiles since \texttt{\small showAndAppend} is now only permitted to be used with arguments of types that are instances of the \haskellIn{Show} type class. Note that, as usual, we do not need to specify the type signature explicitly and the compiler would be able to infer the type \texttt{\small showAndAppend} automatically. In other words, if we just write:
\begin{minted}{haskell}
showAndAppend x = "Value of x is:" ++ show x
\end{minted}
This would still compile, since the compiler would infer the correctly constrained type signature for us. For a type to satisfy a type class constraint, it must implement the interface that is described by the type class. Therefore, if we have a value of some type \haskellIn{a} that is constrained by some type class constraint, then we know that whatever type gets instantiated for \haskellIn{a} will at least support all the functions and operations described by the type class. In order to show that a particular type can satisfy a type class constraint, we must implement a type class instance for it which implements the respective interface.

\makebox[0.5cm]{\faBook}~\emph{Recommended reading}: Chapter 3 of \emph{Learn you a Haskell} \citep{lipovaca2011learn} or Chapters 3 and 8.5 of \emph{Programming in Haskell} \citep{hutton2016programming}.

\makebox[0.5cm]{\faLightbulbO}~Do not forget to test your solutions with \texttt{\small stack test} as you progress through the exercises.

\taskLine 

Functional programming is firmly grounded in mathematics and therefore mathematical abstractions are often useful programming abstractions in Haskell. Abstractions are often represented by type classes. Two examples of this we consider for the following exercises are algebraic structures known as \emph{semigroups} and \emph{monoids}. Type classes for both of these structures exist in Haskell's standard library, but for these exercises we will implement them both ourselves as well as some instances for them.

Such type classes are useful, because they allow us to write functions which generalise possibly many other different functions. For example, the following definition of \haskellIn{mconcat} can generalise \haskellIn{concat}, \haskellIn{sum}, and \haskellIn{product} depending on the \haskellIn{Monoid} instance for \haskellIn{a}. That is, we only need to implement \haskellIn{mconcat} and do not need to implement any of the other, less general functions:

\begin{minted}{haskell}
mconcat :: Monoid a => [a] -> a
mconcat []     = mempty
mconcat (x:xs) = x <> mconcat xs
\end{minted}
For example, when used on lists and integers (assuming a particular implementation of the \haskellIn{Semigroup} and \haskellIn{Monoid} instances for \haskellIn{Int} based on addition) we can see that \haskellIn{mconcat} produces the same results as \haskellIn{concat} and \haskellIn{sum} depending on what type of elements the list it is applied to contains:
\begin{minted}{haskell}
*Lab Lab> mconcat [[True, False], [False]] 
[True, False, False]
*Lab Lab> mconcat [1 :: Int, 2, 3]
6
\end{minted}
Note that for the second example, we needed to annotate one of the literals with \haskellIn{Int} because otherwise their types would be too polymorphic and the Haskell compiler would be unable to figure out that we wish to use the \haskellIn{Int} instance for \haskellIn{Semigroup}.

\subsubsection{Semigroups}

A semigroup is any type which has an associative, binary operation. We can define a type class for this where the binary operator is called \haskellIn{<>}:
\begin{minted}{haskell}
class Semigroup a where 
  (<>) :: a -> a -> a 
\end{minted}
This operator's definition can be \emph{any} binary operation, provided that it is associative. The associativity law for \haskellIn{<>} can be described as follows, although note that there is no way to tell the compiler about this law, show the compiler that it holds, or get the compiler to check that our instances of the \haskellIn{Semigroup} class satisfy it. Later on in the module we will see how to prove by hand that such laws hold:
\begin{displaymath}
\begin{array}{lcrcl}
\textbf{Associativity} & \qquad & x \mappend (y \mappend z) & = & (x \mappend y) \mappend z 
\end{array}
\end{displaymath}
We say that a type \emph{forms} a semigroup if there is an instance of the \haskellIn{Semigroup} type class for it which obeys the associativity law. For some types, there is only one possible instance of the \haskellIn{Semigroup} class, for others there are several, and yet again others have no possible implementations. Haskell only allows us to implement one instance of a given type class for a given type, so for types where there are several possible instances of a given type class, we need to find workarounds (although these are outside the scope of this lab).

\taskLine

\task{Implement \haskellIn{(<>)} of the \haskellIn{Semigroup} instance for \haskellIn{Int} to be some operation that obeys the associativity law. Note that, while you can only give one implementation of \haskellIn{Semigroup} for \haskellIn{Int}, there are at least two possible implementations which satisfy the associativity laws -- can you think of at least two?}

\task{Once you have implemented an instance of \haskellIn{Semigroup} for \haskellIn{Int}, you can use \texttt{\small stack test} to check that it works correctly. If you open the REPL with \texttt{\small stack repl} or \texttt{\small stack ghci} and try to evaluate \texttt{\small 4 <> 8}, you will get a type error warning you of an ambiguous type variable. By asking GHCi for the type of \emph{e.g.} \texttt{\small 4 <> 8}, can you figure out why it is ambiguous? You can resolve the ambiguity by providing an explicit type annotation, \emph{e.g.} by writing \texttt{\small (4 :: Int) <> 8} }

\task{Implement \haskellIn{(<>)} of the \haskellIn{Semigroup} instance for \texttt{\small [a]} so that it implements an operation which obeys the associativity law.}

\task{Once implemented, you can test that your \haskellIn{Semigroup} instance for \texttt{\small [a]} also satisfies the associativity law by running \texttt{\small stack test}. You can also play with the \texttt{\small [a]} instance in the REPL:}
\begin{minted}{haskell}
*Lab Lab> [1,2,3] <> [4,5,6]      
[1,2,3,4,5,6]
*Lab Lab> [True, False] <> [True] 
[True, False, True]
*Lab Lab> "hello" <> "there" 
"hellothere"
\end{minted}

\taskLine 

\subsubsection{Monoids}

A monoid is an algebraic structure which is a semigroup and additionally has a \emph{unit value} (also known as \emph{identity}). In Haskell, we can declare a type class for types which are monoids:
\begin{minted}{haskell}
class Semigroup a => Monoid a where
  mempty :: a
\end{minted}
As we can see, we have a super class constraint which says that in order for some type \haskellIn{a} to be a \haskellIn{Monoid}, it must first be an instance of \haskellIn{Semigroup}. Instances of the \haskellIn{Monoid} type class should obey the following \emph{monoid laws}:
\begin{displaymath}
\begin{array}{lcrcl}
\textbf{Left identity} &\qquad & \mathit{mempty} \mappend x & = & x \\
\textbf{Right identity} &\qquad & x \mappend \mathit{mempty} & = & x 
\end{array}
\end{displaymath}
We say that a type \emph{forms} a monoid if there is an instance of the \haskellIn{Monoid} type class for it which obeys the monoid laws. 

% The \haskellIn{mconcat} function shown above is not necessary for a type to be a monoid, but it generalises the ordinary \haskellIn{concat} function on lists of lists and can easily be implemented with the help of \haskellIn{(<>)} and \haskellIn{mempty}.

% \taskLine 

% \task{The \haskellIn{mconcat} function does nothing specific to any particular type. Specify a default implementation for the \haskellIn{mconcat} function in the declaration of the \haskellIn{Monoid} type class so that it obeys the last monoid law.}

% \task{Does it matter whether you use \haskellIn{foldr} or \haskellIn{foldl} for the implementation of \haskellIn{mconcat}?}

\taskLine

\task{Implement \haskellIn{mempty} of the \haskellIn{Monoid} instance for \haskellIn{Int} so that it obeys the monoid laws. Remember that there are at least two possible implementations for \haskellIn{Semigroup} for \haskellIn{Int} -- how do they affect our choice for the implementation of \haskellIn{mempty}?} 

\task{You can check that your implementation of \haskellIn{mempty} works by running \texttt{\small stack test} now or test out the \haskellIn{mconcat} example shown earlier for a list of integers in the REPL. For example: \haskellIn{mconcat [1 :: Int, 2, 3]}. The result you get will depend on how you chose to implement the \haskellIn{Semigroup} and \haskellIn{Monoid} instances for \haskellIn{Int}.}

\task{Implement \haskellIn{mempty} of the \haskellIn{Monoid} instance for \texttt{\small [a]} so that it obeys the monoid laws. Once you have done so, all the tests should pass when you run \texttt{\small stack test}.}

\taskLine 

The following exercises make use of \emph{higher-order functions}. You may wish to complete the exercises in \Cref{sec:lab-hof} first and then return to the remaining exercises here.

\task{Functions of type \texttt{\small a -> b} form a monoid if \texttt{\small b} is a monoid. Implement the \haskellIn{Semigroup} and \haskellIn{Monoid} instances for \texttt{\small a -> b} so that they obey the semigroup and monoid laws. Note that there are no unit tests for this task as it would require tests for function equality.}

If this seems a bit unintuitive, you can think of this as a way of composing two functions of type \texttt{\small a -> b} to yield a new function of type \texttt{\small a -> b} which takes an input of type \texttt{\small a}, gives it to both original functions which results in two values of type \texttt{\small b} that are then combined into one value of type \texttt{\small b} through the implementation of \haskellIn{(<>)} for \texttt{\small b}. For example:
\begin{minted}{haskell}
*Lab Lab> ((\x -> x ++ [1,2]) <> (\y -> y ++ [3,4])) [5]
[5,1,2,5,3,4]
\end{minted}

\taskLine 