\section{Type-level programming}
\topics{Kinds, phantom types, GADTs, singleton types, pattern matching with GADTs, data type promotion, closed and open type families.}

This final set of exercises is all about type-level programming. The skeleton code can be obtained as usual with the following command:
\begin{minted}{bash}
$ git clone https://github.com/fpclass/lab-tlp
\end{minted}

\faBook~\emph{Further reading}: for more examples of type-level programming in action, \emph{Fun with type functions} \citep{kiselyov2010fun} is a good read. The first two chapters of \emph{Giving Haskell a promotion} \citep{yorgey2012giving} give an overview of key type-level programming techniques in Haskell.

\taskLine 

\task{Define a closed type family \texttt{\small Not} which performs boolean negation at the type-level. Check that your solution works correctly by running \texttt{\small stack test} as usual.}

\task{Once defined, you should be able to use the REPL commands to infer the types and kinds of your new definition. Try the following:}
\begin{itemize}
	\item \texttt{\small :k Not}
	\item \texttt{\small :k Not True}
	\item \texttt{\small :kind!~Not True}
\end{itemize}

\taskLine

\task{Modify the definition of \texttt{\small SBool} to define a singleton type for booleans. This type should have kind \texttt{\small Bool -> *} and two constructors named \haskellIn{STrue} and \haskellIn{SFalse} with appropriate types. Once you have defined this type, verify in the REPL that the type has the correct kind and that the constructors have the correct types. The unit tests will also ensure that \haskellIn{STrue} and \haskellIn{SFalse} have the correct types.}

\task{Having a singleton type for booleans is useful as we can now keep track of the value of a boolean variable at the type-level and therefore at compile-time. This allows us to define functions of types such as:}
\begin{minted}{haskell}
inot :: SBool b -> SBool (Not b)
\end{minted}
That is, given a value of type \texttt{\small SBool b} where \texttt{\small b} is a type of kind \texttt{\small Bool} corresponding to the value, \haskellIn{inot} should return a boolean whose value is the negation of \texttt{\small b}. Implement this function now so that we get the expected behaviour:
\begin{minted}{haskell}
inot STrue  ==> SFalse 
inot SFalse ==> STrue
\end{minted}
While this is not very interesting on the term-level, what happens if you ask the REPL for the types of these expressions?
\begin{itemize}
	\item \texttt{\small :t inot STrue}
	\item \texttt{\small :t inot SFalse}
\end{itemize}

\task{Could you define similar functions for other boolean operations, such as \haskellIn{and}?}

\task{Given a type that is known at compile-time, we may wish to convert it to a corresponding value on the value-level. This process is in general known as \emph{reification} and can be accomplished with the help of a suitable type class. We now want to do this for type-level booleans. You are already given the definition of a suitable type class, named \haskellIn{KnownBool}. Implement suitable instances of this type class so that \haskellIn{boolVal} can be used in the following ways:}
\begin{minted}{haskell}
boolVal (Proxy :: Proxy True)  ==> True
boolVal (Proxy :: Proxy False) ==> False
\end{minted}

\taskLine 

The aim of this last part of these exercises is to implement \emph{heterogeneous lists} in Haskell. The ``ordinary'' lists that we have come across in Haskell are homogeneous: that is, every element has the same type. For example, the following is a valid list in Haskell because all elements have the same type:
\begin{minted}{haskell}
[4,8,15,16,23,42] :: [Int]
\end{minted}  
However, the following is not a valid list in Haskell because its elements have different types:
\begin{minted}{haskell}
[True,"Duck"] -- not well typed
\end{minted} 
This is because lists in Haskell need to be parametrised by the element type: the list type constructor \texttt{\small []} has kind \haskellIn{* -> *}. The definition of lists assumes that every element has that type:
\begin{minted}{haskell}
[]  :: [a]
(:) :: a -> [a] -> [a]
\end{minted} 
In order to implement lists where the elements can have different types, we need to be able to parametrise a list by the types of all of its elements. In other words, we need a type constructor of kind \texttt{\small [*] -> *}. That is, a type constructor which requires a list of types as argument.

\task{With the help of type-level lists, complete the definition of \texttt{\small HList} so that it has two constructors: \haskellIn{HNil} which represents an empty, heterogeneous list and \haskellIn{HCons} which adds an element to a heterogeneous list. Some examples of what should work successfully in the REPL once you are done:}
\begin{minted}{haskell}
*Lab> :t HNil
HNil :: HList '[]
*Lab> :t HCons True HNil
HCons True HNil :: HList '[Bool]
*Lab> :t HCons 4 (HCons True HNil)
HCons 4 (HCons True HNil) :: Num a => HList '[a, Bool]
\end{minted}

\task{Implement the \haskellIn{hhead} function, which should work just like \haskellIn{head} does on ordinary lists, but for heterogeneous lists.}

\task{Define suitable instances of the \haskellIn{Show} type class so that we can use the \haskellIn{show} function on heterogeneous lists. For example:}
\begin{minted}{haskell}
show HNil                          ==> "[]"
show (HCons 4 HNil)                ==> "4 : []"
show (HCons "cake" (HCons 4 HNil)) ==> "\"cake\" : 4 : []"
\end{minted}
Once implemented, all tests should pass.

\task{(\emph{Difficult}) Replace your instances of the \haskellIn{Show} type class that you wrote for the previous exercise with new ones so that the \haskellIn{show} function on heterogeneous lists works as follows instead:}
\begin{minted}{haskell}
show HNil                          ==> "[]"
show (HCons 4 HNil)                ==> "[4]"
show (HCons "cake" (HCons 4 HNil)) ==> "[\"cake\", 4]"
\end{minted}
\emph{Hint}: Constraint kinds\footnote{\url{http://blog.omega-prime.co.uk/2011/09/10/constraint-kinds-for-ghc/}} may help you solve this task.
