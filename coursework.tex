\pagebreak
\section{Coursework (15\% + 25\%)}

There are two pieces of coursework which you will have to complete during Term 2 and submit through Tabula. You will receive feedback for the first coursework before the second coursework is due. Once you have made a submission through Tabula, it will be processed by our custom software called WAAT. After submission, you will have access to WAAT at
\begin{center}
	\url{https://waat.dcs.warwick.ac.uk/}
\end{center}
where you can view all your submissions and the reports that our system has automatically derived from them. You can use this to ensure that your submission compiles and works properly on our systems. 

\subsection{Large Arithmetic Collider (15\%)}

\paragraph{Description} You have to implement a program in Haskell which can solve a challenging combinatorial problem which we have named the \emph{large arithmetic collider}. 

\paragraph{Aims} This coursework is designed to test your ability to write basic Haskell programs using built-in functions, work with lists, and write recursive functions. 

\subsection{Scratch clone (25\%)}

\paragraph{Description} Scratch is a popular tool for teaching programming to children. For this coursework, you have to implement an interpreter for a simple programming language which is used to complete a clone of Scratch. 

\paragraph{Aims} This tests your ability to make use of type classes, data types, and functional design patterns such as monads.